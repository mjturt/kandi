\chapter{Linux ja palvelimet}\label{linux_ja_palvelimet}

\section{Linux}\label{linux}

Linux on perhe käyttöjärjestelmiä, jotka perustuvat Linux käyttöjärjestelmäytimeen eli kerneliin. Linux-kernelin ensimmäisen version julkaisi Linus Torvalds vuonna 1991 opiskellessaan Helsingin Yliopiston Tietojenkäsittelytieteen laitoksella. Linux-käyttöjärjestelmällä viitataan mihin tahansa Linux-kerneliä käyttävään käyttöjärjestelmään. Linux on UNIX-tyyppinen käyttöjärjestelmä, mutta ei jaa samaa koodikantaa UNIX-käyttöjärjestelmien kanssa, ei ole sertifioitu eikä noudata Single UNIX Specification -standardia. Linux-kerneliä käyttävät käyttöjärjestelmät paketoidaan yleensä Linux-jakeluksi, jotka useimmiten sisältävät kernelin lisäksi kokoilman ohjelmistoja sekä paketinhallinnan. Linux-kerneli ja useimmat Linux-jakelut ovat vapaata lähdekoodia.~\cite{openbookos}

Työpöytäkäytössä Microsoft Windows on suosituin käyttöjärjestelmä, Linuxia käyttävät vuonna 2020 vain 1,53\% työpöytäkäyttäjistä~\cite{statcounter}. Mobiililaitteissa suosituin käyttöjärjestelmä on Googlen Android, jonka voi katsoa olevan Linux-jakelu sen käyttäessä muokattua versiota Linux-kernelistä.~\cite{google_android} Mobiililaitteista Androidia käytti 85 \% vuonna 2018~\cite{statcounter}. Vuosina 2017–2019 maailman 500 tehokkaimmasta supertietokoneesta kaikki käyttävät Linuxia.~\cite{itsfoss}. Internetin julkisista web-palvelimista vuonna 2015 Linuxia käyttää 71,6\%-96,4\% riippuen lähteestä.~\cite{w3techs}~\cite{w3cook}

Linux-jakeluiden kotisivujen kävijämäärien perusteella suosituimmat 3 Linux-jakelua ovat MX Linux, Manjaro ja Mint (21.10.2020).~\cite{distrowatch} Useimpien Linux-jakeluiden vapaan saatavuuden vuoksi on vaikea arvioida todellisia käyttäjämääriä, mutta suosituimpia Linux-jakeluita palvelinkäytössä lienevät Red Hat Enterprise Linux, SuSE, Ubuntu, Debian sekä CentOS.~\cite{linuxcom}

Perusteita suurelle Linuxin käytölle palvelimissa on arvioitu olevan vakaus ja luotettavuus, turvallisuus, muokattavuus, lähdekoodin avoimuus sekä kustannukset.~\cite{linuxjournal}

\section{Palvelimet}\label{palvelimet}

Palvelin on tietokonejärjestelmä, joka tarjoilee palveluja, dataa tai muita resursseja asiakastietokoneilleen tai –sovelluksilleen, useimmiten internetin välityksellä. Palvelimet koostuvat yleensä palvelinkäyttöön tarkoitetusta tietokoneesta sekä palvelinkäyttöön tarkoitetusta käyttöjärjestelmästä. Erityyppisiä palvelimia ovat mm.\ tiedostopalvelimet, tulostinpalvelimet, sovelluspalvelimet, DNS-palvelimet, sähköpostipalvelimet, tietokantapalvelimet ja web-palvelimet.

Järjestelmän arkkitehtuuria, jossa palvelin palvelee asiakaskonetta, kutsutaan asiakas–palvelin malliksi. Tyypillisesti palvelimet ja asiakkaat keskustelevat keskenään pyyntö– ja vastausperiaatteella. Asiakas lähettää pyynnön palvelimelle, johon palvelin vastaa. Esimerkiksi asiakkaan web-selain lähettää HTTP-pyynnön palvelimelle, johon palvelin vastaa HTML:n muodossa.

Käytännössä mikä tahansa tietokone voi olla palvelin, mutta yleensä palvelimet ovat palvelinkäyttöön tarkoitettuja tietokoneita, jotka sijaitsevat palvelinsalissa. Palvelintietokoneet koostuvat osista, jotka ovat luotettavampia kuin kuluttajatietokoneissa. Useimmiten palvelintietokoneet ovat räkkiin sopivassa vaakamallisessa kotelossa ja räkissä palvelimia voi olla useita kymmeniä. Suurimmissa palvelinsaleissa voi olla satoja räkkejä. Palvelin ei tarvitse näyttöä tai syöttölaitteita kuten näppäimistöä tai hiirtä muuta kuin huoltotoimenpiteissä, joten tilan ja kustannusten säästämiseksi näitä harvemmin on palvelimissa. Palvelinta kontrolloidaan etänä esimerkiksi SSH:n välityksellä, web-pohjaisesta käyttöliittymästä tai jollakin kaupallisella ratkaisulla kuten Microsoft Managment Consolella.

Palvelintietokoneissa osien kokoonpano pyrkii mahdollisemman suureen toimintavarmuuteen. Tekniikoita joilla toimintavarmuutta pyritään takaamaan ovat muun muassa virheenkorjaava muisti (ECC), osien lennosta vaihto, kriittisten osien tuplana saatavilla oleminen, RAID–levyjärjestelmät sekä virransyötön takaaminen akustolla (UPS) tai jopa generaattoreilla. Tyypillinen palvelin pysyy toimintakykyisenä vaikka siitä hajoaisi virtalähde tai tallennuslaite kuten kovalevy tai vaikka koko rakennuksesta katkeaisivat sähköt.~\cite{paessler}

Palvelin voi olla myös toisen palvelimen tarjoama virtuaalipalvelin. Tässä tapauksessa fyysinen palvelin toimii virtuaalipalvelinalustana ja voi ylläpitää useita kymmeniä virtualisoituja käyttöjärjestelmiä. Nykyisin virtuaalipalvelinalusta koostuu useista fyysisistä palvelimista tai jopa palvelinsaleista ja resursseja pystyy allokoimaan virtuaalipalvelimille joustavasti (klusterointi).

Käytännössä varsinaisten fyysisten palvelinten ja palvelinsalien ylläpito on keskittynyt muutamille suurille palveluntarjoajille, joilla on käytössään useita kymmeniä palvelinsaleja. Harvat palvelinresursseja tarvitsevat ylläpitävät itse omia fyysisiä palvelimiaan omissa tiloissaan. Tyypillisesti resurssit vuokrataan palveluntarjoajalta. Palvelinresurssit voivat olla virtuaalipalvelimia, fyysisiä palvelimia palveluntarjoajan tiloissa tai pääsy yhteisessä käytössä olevalle palvelimelle.~\cite{golden2011virtualization}

\section{Tyypillinen Linux-palvelinkonfiguraatio}\label{tyypillinen_palvelinkonfiguraatio}

    Esittelen seuraavaksi kuvitteellisen, mutta realistisen esimerkin palvelinarkkitehtuurin toteutuksesta laitteistosta ohjelmistoihin, loppukäyttäjästä ylläpitoon. Päämääränä on tarjota palvelinresurssit keskisuuren yrityksen web-sovellukselle.

    Yritys vuokraa palveluntarjoajalta virtuaalipalvelimen. Palveluntarjoajalla on useita suuria palvelinsaleja. Kyseisen virtuaalipalvelimet tarjoillaan yhdestä palvelinsalista jossa on 100 kpl räkkejä, joissa jokaisessa on 10 palvelintietokonetta. Yksittäisen räkin varavirtalähteenä on akusto, joka sijaitsee räkin alaosassa. Koko palvelinsalin varavirranlähteenä toimii diesel-aggregaatti. Palvelinsalin palvelintietokoneista on allokoitu virtuaalipalvelinten vuokraamiseen 10 räkin eli 100 palvelintietokoneen verran. Palvelintietokoneet ovat identtisiä keskenään. Suuren kapasiteetin tarpeen vuoksi niissä on useampi prosessori sekä runsaasti virheenkorjaavaa keskusmuistia. Tallennustilana toimii 10 SSD-levyä. Levyt ovat kytketty RAID 6 järjestelmään tarjoten näin tallennustilaa 80\% levykapasiteetista kahden levyn redundanssilla. Levyt on lennosta vaihdettavia, joten levyn rikkoontuessa palvelimen toiminta ei keskeydy. Yhdessä palvelinkoneessa on 2 lennosta vaihdettavaa virtalähdettä.

    Virtuaalipalvelinalustat käyttävät Red Hat Enterprise Linuxia käyttöjärjestelmänään. Virtualisointiin käytetään vapaan lähdekoodin QEMU–projektia. Virtuaalikoneita on keskimäärin 10 yhdellä fyysisellä palvelimella.

    Asiakasyritys vuokraa yhden virtuaalipalvelimen, jolle allokoidaan 1/10 fyysisen palvelimen resursseista. Virtuaalipalvelimen käyttöjärjestelmänä on Ubuntu Linux. Virtuaalipalvelinta ohjataan SSH-yhteiden välityksellä. Yriyksen web-sovellus on Python–ohjelmointikielellä kirjoitettu. Web-sovelluskirjastona on käytetty Djangoa. Djangon sisäinen HTTP-palvelinsovellus tarjoilee sisällön ainoastaan IP:lle 127.0.0.1 porttiin 3000. Samalla virtuaalipalvelimella ajetaan myös Nginx–nimistä HTTP-palvelinta, joka toimii käänteisenä välityspalvelimena paikallisen HTTP-palvelimen ja ulkoverkon välillä. Nginx HTTP-palvelin välittää Django-palvelimen portin 3000 ulkoverkkoon portteihin 80 ja 443.

    Palvelinresursseja vuokraava yritys tarjoaa myös DNS-nimipalveluita. Virtuaalipalvelimen IP:lle allokoidaan esimerkki.fi domain.

    Asiakasyrityksen web-sovellus on nyt saatavilla HTTP (S) -protokollan ylitse esimerkki.fi osoitteesta portista 80 tai 443. Asiakasyrityksen asiakkaat vierailevat web-selaimellaan osoitteessa esimerkki.fi. Asiakkaan tietokone lähettää ensin DNS-tiedustelun ja saa vastaukseksi virtuaalipalvelimen IP-osoitteen. Tämän jälkeen web-selain lähettää HTTP–pyynnön kyseiseen IP:seen porttiin 80. Virtuaalipalvelimella pyörivä Nginx välittää pyynnön Djangon HTTP-palvelimelle, joka vastaa pyyntöön HTML-koodilla. Nginx välittää tämän HTML:n takaisin asiakkaan web-selaimelle ja web-selain renderöi HTML-koodista web-sivuston.
