\chapter{Yhteenveto}\label{yhteenveto}

Tarkastelun kohteena oli muutamia yleisimpiä kyberhyökkäyksiä ja keinoja vahventaa Linuxin tietoturvaa. Esitellyt keinot vahventaa Linux-palvelinten tietoturvaa tarjoavat vähintään vähäistä suojaa esitellyiltä kyberhyökkäyksiltä.

Taulukkoon~\ref{tab:security-table} on koottu käsitellyt hyökkäykset suhteessa suojautumismenetelmiin. Iso \textbf{X} tarkoittaa sitä, että suojautumismenetelmä auttaa merkittävästi suojautumaan tietoturvahyökkäykseltä. Pieni \textbf{x} tarkoittaa puolestaan sitä, että menetelmä auttaa suojautumaan hieman tietoturvahyökkäykseltä. Pienen \textbf{x}:n merkintää käytetään myös silloin, jos suojautumismenetelmä auttaa merkittävästi vähentämään tietoturvahyökkäyksen aiheuttamaa vahinkoa.

Kuten taulukosta~\ref{tab:security-table} huomaa, esitellyt suojautumismenetelmät tuntuvat jakautuvan kahden pääryhmän välillä. Ensimmäisen ryhmän suojautumismenetelmät auttavat tehokkaasti, mutta vain yhteen tiettyyn kyberhyökkäykseen. Kun taas toisen ryhmän menetelmät ovat kokonaisvaltaisempia ratkaisuja, jotka auttavat monilla osa-alueilla hieman.

Taulukosta~\ref{tab:security-table} huomataan myös, että joltakin hyökkäyksiltä on hankala suojautua. Tällainen hyökkäys on esimerkiksi palvelunestohyökkäys~(\ref{dos}). Hyökkäykseltä~\ref{dos} pystytään palomuurien (\ref{palomuurit}) avulla suojautumaan vain mikäli hyökkäys tulee yhdestä lähteestä, ennalta arvattavista lähteistä tai muutoin ennalta-arvattavalla tavalla.

\begin{table}
\centering{}\caption{Hyökkäykset vs. suojautumismenetelmät\label{tab:security-table}}
\begin{tabular}{c|c|c|c|c|c|c|c|}
   &\ref{dos}&\ref{backdoors}&\ref{verkon_kuuntelu}&\ref{privilege_escalation}&\ref{injection}&\ref{bruteforce}&\ref{theft} \tabularnewline\hline
    Tallennustilan salaaminen~(\ref{tallennustilan_salaaminen}) & & & & & & & \textbf{X}\tabularnewline\hline
    Tietoliikenteen salaaminen~(\ref{tietoliikenteen_salaaminen}) & & & \textbf{X} & & & &\tabularnewline\hline
    Palomuurit~(\ref{palomuurit}) & \textbf{x} & & & & \textbf{x} & \textbf{x} &\tabularnewline\hline
    Eristys ja virtualisointi~(\ref{eristys_ja_virtualisointi}) & & \textbf{x} & & \textbf{x} & \textbf{x} & \textbf{x} &\tabularnewline\hline
    Autentikaatioprosessin vahvennus~(\ref{autentikaatio}) & & & & & & \textbf{X} &\tabularnewline\hline
    Käyttöoikeuksien vahvennus~(\ref{access_rights}) & & \textbf{x} & & \textbf{x} & \textbf{x} & &\tabularnewline\hline
\end{tabular}
\end{table}
