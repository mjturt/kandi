\chapter{Johdanto}\label{Johdanto}

    Palvelimen vaarantunut tietoturva voi asettaa alttiiksi tuhansien, jopa miljoonien ihmisten tietoja. Liiketoiminnan kontekstissa palvelimen tietoturvan pettäminen voi johtaa miljoonien eurojen menetykseen liiketoimintakriittisen sovelluksen ollessa pois käytöstä. Pahimmillaan voidaan puhua ihmishenkien menetyksen vaarasta kun kyseessä on esimerkiksi terveydenhuollon infrastruktuuriin kuuluva palvelin. Palvelinten tietoturvaa voi edellä mainittujen seikkojen vuoksi pitää huomattavasti tärkeämpänä kuin esimerkiksi tavallisen työpöytätietokoneen.

    Palvelinten tietoturva on tärkeää myös palvelinten luonteen vuoksi. Palvelimen on oltava helposti saatavilla asiakassovelluksilleen. Palvelimen suora saatavuus internetissä ja palvelut, joita palvelin ajaa, luovat palvelimen hyökkäyspinta-alasta korkeamman kuin esimerkiksi tavallisen työpöytätietokoneen.

    Valtaosa palvelimista käyttää käyttöjärjestelmänään Linuxia. W3Cookin analyysin mukaan jopa 96,4 \% julkisista web-palvelimista käyttää Linux-käyttöjärjestelmää\cite{w3cook}. Tästä syystä tutkielma keskittyy nimenomaan Linux-palvelinten tietoturvaan.

    Tutkielmani haasteena on esitellä kuinka Linux-palvelimen ylläpitäjä voi suojautua tyypillisimmiltä tietoturvauhilta. Toisessa luvussa johdattelen Linuxin ja palvelinten perusteisiin, jonka jälkeen luvussa 3 käyn läpi tärkeimpiä konsepteja tietoturvasta puhuttaessa ja tyypillisimpiä tietoturvauhkia sekä kyberhyökkäyksiä. Luvussa 4 esittelen kuinka näiltä tietoturvauhilta ja kyberhyökkäyksiltä voi suojautua.
