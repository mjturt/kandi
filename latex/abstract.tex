\keywords{Tietoturva, Linux, Palvelin}
\begin{abstract}
    Miten Linux-palvelinylläpitäjä voi suojautua kyberhyökkäyksiltä? Tutkielman haasteena on tarkastella keinoja suojautua tyypillisimmiltä kyberhyökkäyksiltä Linux-palvelinympäristössä.

    Palvelimeen kohdistuneen kyberhyökkäyksen seuraukset voivat olla erittäin tuhoisat. Tyypillisesti seuraukset ovat palveluntarjoajille rahallisia ja palveluiden käyttäjille palvelun käyttökatkoja, mutta erityisen tuhoisassa hyökkäyksessä voidaan puhua ihmisten terveyteen kohdistuvasta vaarasta.

    Linux on yleisin käyttöjärjestelmä palvelimissa. Tästä syystä tutkielma tarkastelee keinoja suojautua kyberhyökkäyksiltä nimenomaan Linux-palvelinten näkökulmasta.

    Tutkielmassa esitellään muutamia yleisimpiä kyberhyökkäyksiä ja millaisia resursseja nämä uhkaavat. Tämän jälkeen esitellään muutamia yleisimpiä menetelmiä vahventaa tietoturvaa Linux-palvelimilla ja miten ne auttavat suojautumaan tutkielmassa esitellyiltä kyberhyökkäyksiltä. Menetelmiä suojautua kyberhyökkäyksiltä käsitellään ensin yleisellä tasolla, jonka jälkeen käydään käytännönläheisesti läpi miten menetelmää voi soveltaa Linux-palvelinympäristössä.
\end{abstract}
